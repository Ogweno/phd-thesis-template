%!TEX root = ../thesis.tex
%*******************************************************************************
%*********************************** First Chapter *****************************
%*******************************************************************************

\chapter{Εισαγωγή}  %Title of the First Chapter

\ifpdf
    \graphicspath{{Chapter1/Figs/Raster/}{Chapter1/Figs/PDF/}{Chapter1/Figs/}}
\else
    \graphicspath{{Chapter1/Figs/Vector/}{Chapter1/Figs/}}
\fi

Ένα αρχείο κλάσης του \LaTeX \index{Latex} περιέχει όλα τα στοιχεία διαμόρφωσης του προτύπου. Το πρότυπο NTUA έχει ως βάση το πρότυπο CUED (Cambridge University - Engineering Department) \url{https://github.com/kks32/phd-thesis-template}. Ακολουθεί όμως όλες τις προδιαγραφές σύμφωνα με τις αποφάσεις της Συγκλήτου του ΕΜΠ.

Στο παρόν κείμενο έχουν χρησιμοποιηθεί κάποια παραδείγματα χρησης των Ελληνικών, πινάκων, εικόνων, ενώ κάποια στοιχεία έχουν μείνει όπως στο πρωτότυπο κείμενο στα Αγγλικά (δε θα ταν λίγο γελοίο να κάνω μεταφράσεις αχρείαστες;;!!!).

Το παρόν πρότυπο διατείθεται ελεύθερα από τη διεύθυνση \url{https://github.com/demanasta/phd-thesis-template} υπό την άδεια ΜΙΤ. Σε περίπτωση που υπάρχει κάποιο πρόβλημα με το παρόν έγγραφο μπορείτε να επικοινωνήσετε με τον Δημήτρη στο: \href{mailto:dganastasiou@gmail.com}{dganastasiou@gmail.com} ή ανοίγοντας ένα νέο αίτημα/θέμα στο \url{https://github.com/demanasta/phd-thesis-template/}

Καλή επιτυχία σε όποιον το χρησιμοποιήσει, το διαμορφώσει για το δικό του πανεπιστήμιο. Κάθε ιδέα για βελτίωση και διόρθωση καλαοδεχούμενη!

\nomenclature[z-DEM]{DEM}{Discrete Element Method}
\nomenclature[z-FEM]{FEM}{Finite Element Method}
\nomenclature[z-PFEM]{PFEM}{Particle Finite Element Method}
\nomenclature[z-FVM]{FVM}{Finite Volume Method}
\nomenclature[z-BEM]{BEM}{Boundary Element Method}
\nomenclature[z-MPM]{MPM}{Material Point Method}
\nomenclature[z-LBM]{LBM}{Lattice Boltzmann Method}
\nomenclature[z-MRT]{MRT}{Multi-Relaxation 
Time}
\nomenclature[z-RVE]{RVE}{Representative Elemental Volume}
\nomenclature[z-GPU]{GPU}{Graphics Processing Unit}
\nomenclature[z-SH]{SH}{Savage Hutter}
\nomenclature[z-CFD]{CFD}{Computational Fluid Dynamics}
\nomenclature[z-LES]{LES}{Large Eddy Simulation}
\nomenclature[z-FLOP]{FLOP}{Floating Point Operations}
\nomenclature[z-ALU]{ALU}{Arithmetic Logic Unit}
\nomenclature[z-FPU]{FPU}{Floating Point Unit}
\nomenclature[z-SM]{SM}{Streaming Multiprocessors}
\nomenclature[z-PCI]{PCI}{Peripheral Component Interconnect}
\nomenclature[z-CK]{CK}{Carman - Kozeny}
\nomenclature[z-CD]{CD}{Contact Dynamics}
\nomenclature[z-DNS]{DNS}{Direct Numerical Simulation}
\nomenclature[z-EFG]{EFG}{Element-Free Galerkin}
\nomenclature[z-PIC]{PIC}{Particle-in-cell}
\nomenclature[z-USF]{USF}{Update Stress First}
\nomenclature[z-USL]{USL}{Update Stress Last}
\nomenclature[s-crit]{crit}{Critical state}
\nomenclature[z-DKT]{DKT}{Draft Kiss Tumble}
\nomenclature[z-PPC]{PPC}{Particles per cell}