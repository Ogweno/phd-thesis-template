% ************************** Thesis Abstract *****************************
% Use `abstract' as an option in the document class to print only the titlepage and the abstract.
%\begin{otherlanguage}{english}
\begin{extsum}

\graphicspath{{Chapter4/Figs/Vector/}{Chapter4/Figs/}{Chapter5/Figs/Vector/}{Chapter6/Figs/Vector1/}{Chapter7/Figs/Vector/}}

  \begin{center}
  	{\normalsize \underline{Extented Summary of Doctoral Dissertation (e-version)} \par}
    { \Large {\bfseries {Terrestrial and satellite geodetic data analysis for estimation of crustal deformation}} \par}
    {\large \vspace*{1em} {\bfseries {Demitris G. Anastasiou}} \par}
    {\normalsize {Dipl. Rural \& Surveying Engineer NTUA} \par}
%    {\normalsize \vspace*{1em} {The dissertation submitted for the degree of} \par}
%    {\large {\bfseries {Doctor of Engineering}} \par}
  \end{center}

  \let\thefootnote\relax\footnotetext{\looparrowright The dissertation accepted from School of Rural and Surveying Engineering (SRSE) of National Technical University of Athens (NTUA) on July 2017 for the degree of \textbf{Doctor of Engineering}.}
  \let\thefootnote\relax\footnotetext{\looparrowright Download the main document in Greek from \href{https://pithos.okeanos.grnet.gr/public/aGFMHJ6v8Qhmd2s9WwE9A1}{here}.}
  \let\thefootnote\relax\footnotetext{\looparrowright Contact to Demitris Anastasiou via \href{mailto:dganastasiou@gmail.com}{mail} or \href{https://www.linkedin.com/in/demitrisanastasiou/}{LinkedIn}.}


Greece is located on the boundary of convergence of three tectonic plates, making it the most active tectonic and seismic area in Europe. The complexity of the tectonic background makes the region a natural laboratory, study of tectonic phenomena such as strong earthquakes and intense deformations. As a result, there is a lot of scientific work that has been carried out to better identify the deformation field in Greece.

The research focuses on the regions of Western Greece and Central Ionian Sea as well as Central Greece and the Corinthian Gulf. The areas were selected as they consist of zones of intense tectonic and seismic activity, while densely populated cities and major infrastructure are located there. Furthemore, the first angular measurements of the trigonometric network were carried out during the period 1889-1893 on these regions.

A trigonometric network is an important background for the implementation of a reference system for mapping, cadastre, surveying etc.. In Greece, the first effort for a National network began in the middle of the 19th century. In 1889, the trigonometric network re-established and is the one used up to now. Trigonometric stations can also be used as control points to identify displacements in a region. 
Taking advantage of such historical measurements requires an additional process of catalogueing, evaluating and archiving the data. The methodology followed here plays a significant part in this study. The terrestrial geodetic data were collected from the historical archives of Hellenic Militery Geografical Service (HMGS). A database was designed and implemented for the first time in order to collect, ecaluate and store all the available information (Figure \ref{fig_e:triangledb} [\ref{fig:triangledb}]* \footnote{* The numbers on the square brackets refer to the figures and tables on the main document.}). The database is structed in such a way that new data can be included.

In the last 30 years campaign GPS measurements were carried out at 31 points between 1989 and 2001 in order to analyse the kinematic behavior of Western Greece (Table \ref{tab_e:ioncsta} [\ref{tab:ioncsta}], Figure \ref{fig_e:patraNET} [\ref{fig:patraNET}]). Tectonic velocities of 19 permanent GNSS stations located in the area were also analyzed. Furthermore, 4 stations used by \citet{Hollenstein2008} were also evaluated (Table \ref{tab_e:ionpsta} [\ref{tab:ionpsta}], Figure \ref{fig_e:ionioNET} [\ref{fig:ionioNET}]).

The terrestrial geodetic measurements analysis, used 56 triangles which were selected from the trigonometric network covering the region (Figure \ref{fig_e:trianglesALL} [\ref{fig:trianglesALL}]). The data has been divided into three time periods while, the analysis was carried pairs of epochs of obervations (Table \ref{tab_e:directions_full} [\ref{tab:directions_full}]).The results include shear strain and direction of elongation axes for each triangle as they were calculated by the \citet{Frank1966} method.
% \textcolor{red}{ Also, a timeless analysis of changes in the deformation field the first to the last period of analysis presented, as well as the comparison with the results of available scientific works.}

The study then focuses on the kinematic analysis of Western Greece, using satellite GNSS data. Based on the tectonic velocities of the stations, the region has been subdivided into blocks and the parmaeters of the strain tensors are estimated. Models of 4 blocks for the wider ​​Western Greece region and of 5 blocks for the Central Ionian and Patras Gulf were tested. Also, the results for the triangles formed by the GNSS stations and the respective strain tensors derived from the terrestrial abservations are compared.

The last part of the thesis investigates the contribution of geodetic data to the analysis of seismic events. Six strong seismic events were selected in the region of Western Greece for the time period between the years 1909 to 1965 (Table \ref{tab_e:fm1900} [\ref{tab:fm1900}]). The Coulomb stress field is calculated using as input parameters of the earthquake faults the seismic ones. The displacement field (episodic) compared with the results from the terrestrial geodetic data (secular displacement field). The research focuses on triangles with high values ​​of total shear strain and investigates whether these values ​​are due to seismic activity or gross errors in the observations.

For the recent years, the Kephalonia's seismic sequence was analyzed. On January 26, 2014 a strong earthquake of magnitude Mw=6.0 occurred on Kephalonia Island followed by a similar magnitude earthquake Mw=5.9 one week later on February 3, 2014. Extensive structural damages, landslides and many damages on the islands main roads, harbour and airport occured mainly on the western and central part of the island. Significant shifts have been recorded in the GNSS permanent stations located on the island of Kefalonia. Using different models for fault geometry as suggested by previous research (Tables \ref{tab_e:cmt2601} [\ref{tab:cmt2601}], \ref{tab_e:cmt0302} [\ref{tab:cmt0302}]), surface displacements were calculated using the \citet{Okada1985} algorithm and were compared with the shifts of GNSS permanent stations.

The final conclusions  of the kinematic behavior for the region of interest and the possible correlation with strong seismic events are presented below.

% % conclutions..........
The analysis of the terrestrial geodetic data aggrees with the \citet{stiros1993283} results, that the the deformation field before and after 1930 differs, which is expected as similar data have been used. For the first period of analysis, the total shear rate is of the order of 80 nstrain/y. In the second period, after 1930, the total shear strain increases. The values ​for the third period of analysis are similar, while some regions show very high rates of total shear strain, which may be related to seismic activity in the Corinthian Gulf region or, alternatively due to possible presence of gross errors in the measurements. Also, for the region of ​​the Central Ionian and Patraikos Gulf, a change in the direction of the total shear from ESE to WNW, parallel to the direction of the main faults in the region is observed (Figure \ref{fig_e:shear_ionio} [\ref{fig:shear_ionio}]).

In the region of Western Greece and the Ionian Sea, the direction of the principal axes of elongation is N - S in all three periods of analysis (Figure \ref{fig_e:ext_ionio} [\ref{fig:ext_ionio}]). This is consistent with corresponding work which used terrestrial data \citep{Veis1992, Billiris1991, Davies199724571, stiros1993283}. It also coincides with the results of recent studies for the deformation field in this region using GNSS data \citep{Balodimou1994, Chousianitis2015, Floyd2010, Ganas201362, Hollenstein2008, Kahle199541, Perouse2012}.

For the region of Sterea Hellas many trigonometric stations were re-established, especially in the central part, so there were not many compatible observations for the analysis. This did not allowed a detailed analysis of the shear strain field in this region. From the second period of analysis, elongation is observed in NE - SW direction (Figure \ref{fig_e:ext_sterea} [\ref{fig:ext_sterea}], \ref{fig_e:shear_sterea} [\ref{fig:shear_sterea}]) which is consistent with the results of the works \citet{Veis1992, Davies199724571, Marinou2014}. In the third period of analysis, including the sequence of earthquakes in the Alkyonides Islands, it is observed that the wider area of ​​the Kaparelli and Erythres faults shows a higher rate of change of the total shear strain and its direction is parallel to the main faults of the region (Figure \ref{fig_e:sterea0301} [\ref{fig:sterea0301}]). Also the elongation axis is in the NNW - SSE direction, inconsistent with the general kinematics of the area. However, these results are consistent with the deformation field presented in the work of \citet{Marinou2015a} for the area of ​​the Kaparelli fault.

The analysis of the GNSS geodetic data was concentrated in the region of Western Greece. The area of ​​northwest Greece is moving at a very low velocity with respect to a stable Europe in a NNW direction, while it is extending to the NNW - SSE with a similarly low rate (Figures  \ref{fig_e:ionioVEL} [\ref{fig:ionioVEL}], \ref{fig_e:patraVEL} [\ref{fig:patraVEL}]). South of Lefkada Island, velocities are increasing in SW direction (Figure \ref{fig_e:ionioVEL} [\ref{fig:ionioVEL}]). A zone where the velocities south of it remain constant (Figure \ref{fig_e:ptr05bSTR} [\ref{fig:ptr05bSTR}]) is observed in Northwest Peloponnesus, as in the papers of \citet{Perouse2013,Floyd2010}.

From the interpretation of the strain tensors it appears that the region of Central Greece - Ionian Sea extends in a N - S direction, while the southern area of ​​the Ionian in a NNE - SSW one (Figures \ref{fig_e:iontrSTRcmt} [\ref{fig:iontrSTRcmt}], \ref{fig_e:ion4bSTRcmt} [\ref{fig:ion4bSTRcmt}]). In the wider area of ​​the Central Ionian, a single segment is rotated clockwise at a rate of 7.439 ± 0.973 deg/My (Figures \ref{fig_e:ion4bROT} [\ref{fig:ion4bROT}], \ref{fig_e:ptr05bROT} [\ref{fig:ptr05bROT}]). This is in agreement with the studies of \citet{Perouse2013} and \citet{Chousianitis2015} for the kinematic analysis of the region. However, a dense GNSS network would help to detect the boundaries of this region more clearly. The area of ​​the Gulf of Patras and the western part of the Corinthian Gulf is extending to the NNE - SSW with strain rates of the order of 411 and 293 nstrain/y (Figure \ref{fig_e:ptr05bSTR} [\ref{fig:ptr05bSTR}]).

Comparing the analysis between terrestrial measurements from 1890 to 1987 and the GNSS data over the last 25 years, it seems that the Ionian region, with small deviations, extends to the N - S direction for the entire time period of geodetic observations (Figures \ref{fig_e:ext_ionio} [\ref{fig:ext_ionio}], \ref{fig_e:ciontrSTR} [\ref{fig:ciontrSTR}]). In ​​the Gulf of Patras, the deviations of the elongation axis are larger, between past and present, while it seems the direction chenges from NNW - SSE to NNE - SSW (Figures \ref{fig_e:ext_ionio} [\ref{fig:ext_ionio}], \ref{fig_e:ciontrSTR} [\ref{fig:ciontrSTR}]).

The analysis of the six strong seismic events in the region of Western Greece, show that some triangles with higher total shear values are located in the area where a strong earthquake occured. In the care of the earthquakes of Katuna fault in 1921 (Figure \ref{fig_e:kat_1921} [\ref{fig:kat_1921}]), Killini in 1939 (Figure \ref{fig_e:kyl_1939} [\ref{fig:kyl_1939}]) and Eliki in 1965 (Figure \ref{fig_e:eli_1965} [\ref{fig:eli_1965}]), the high values observed are likely to be due to gross errors in measurements rather than the seismic events.

The earthquakes of Nafpaktos in 1909 (Figure \ref{fig_e:naf_1909} [\ref{fig:naf_1909}]) and Katouna in 1953 (Figure \ref{fig_e:kat_1953} [\ref{fig:kat_1953}]) may have affected the adjacent trigonometric stations. The strongest earthquake of Kefalonia in 1953 (Figure \ref{fig_e:kef_1953} [\ref{fig:kef_1953}]) has clearly contributed to the enormous values ​​of the strain tensor estimated of the adjacent triangles. 

The analysis of the recent earthquake sequence in Kefalonia (January 26 - February 3, 2014) (Figures \ref{fig_e:tsallkefa} [\ref{fig:tsallkefa}], \ref{fig_e:ts_1m} [\ref{fig:ts_1m}], \ref{fig_e:tssep} [\ref{fig:tssep}]) shows significant shifts at the permanent stations on Kefalonia Island (Figures \ref{fig_e:disp_split} [\ref{fig:disp_split}], \ref{fig_e:disp_merge} [\ref{fig:disp_merge}]).The displacements of  KEFA station in Lixouri, of the order of 52.0 mm for the first earthquake and 96.0 mm for the second earthquake in a SSE direction, are significantly larger than the ones observed for the other stations (Table \ref{tab_e:kefshift} [\ref{tab:kefshift}]). This difference may be due either to the fact that the epicenter of the earthquakes is very close to the site of the station or due to the liquefaction phenomena observed in the area \citep{Valkaniotis2014,Papadopoulos2014}, as attributed by \citet{Ganas2015}. Also, the opposite direction of the KEFA (SSE) and KIPO (NNW) shifts confirms the results of \citet{Sokos2015, Sakkas2015, Ganas2015, Karakostas2015} that the zone of activated fault is located between the two stations on the peninsula of Paliki.

For the first seismic event, from the investigation of the fault geometry, the solutions converge to a fault with a NNE direction and a slope to the east. In the solution of \citet{Ganas2015} (Figure \ref{fig_e:inp_260101} [\ref{fig:inp_260101}]), the shifts of the stations from the \citet{Okada1985} model  are in line with those of the geodetic data, with the exception of station KEFA. For the second seismic event, the shifts of the GNSS stations are more in agreement with the \citet{Okada1985} surface deformation field for a two-segment fault as suggested by \citet{Boncori2015} (Figure \ref{fig_e:inp_030203} [\ref{fig:inp_030203}]). Compatibility is expected since this model has used both GNSS and InSAR data.

% ***************** FIGURES - TABLE ************************************************ 
\chapter*{Tables and Figures}
\newpage
% ***************** CHAPTER 4 ******************************************************

\begin{table}[H]
  \caption{Triangulation stations and directions recorded for each observation epoch.}
  \label{tab_e:directions_full}

  \begin{center}
    \begin{tabular}{c c c c}
      \toprule
        Epoch code & Time period & Stations & Observations \\
      \midrule
        9010 & 1889 - 1910 & 93 & 251 \\
        1940 & 1919 - 1940 & 113 & 535 \\
        4968 & 1949 - 1968 & 137 & 785 \\
        6987 & 1969 - 1987 & 137 & 823 \\
      \midrule
      \multicolumn{3}{r|}{Sum of Observations} & {2394} \\
      \bottomrule
    \end{tabular}
  \end{center}
\end{table}

\begin{figure}[H]
  \begin{center}
    \includegraphics[width=\linewidth]{triangledb.png}\par
%     \caption{Διάρθρωση της βάσης δεδομένων για την αρχειοθέτηση των επίγειων μετρήσεων.}
     \caption{Structure of the database for archiving terrestrial measurements.}
    \label{fig_e:triangledb}
  \end{center}
\end{figure}

\begin{table}[H]{\small
    \caption{Permanent GNSS stations for the Ionian Sea.}
    \label{tab_e:ionpsta}
     \begin{center}
      \begin{tabular*}{\linewidth}{@{\extracolsep{\fill}}l c c r l c c }
\toprule
Κωδ. & φ ($^{\circ}$) & λ ($^{\circ}$) & \multicolumn{1}{c}{h (m)} & Location & Network & Time period \\ 
% & \multicolumn{2}{c}{(deg)} & \multicolumn{1}{c}{(m)} & \multicolumn{6}{c}{(mm/y)} \\ 

\midrule
AGRI & 38.6235 & 21.4078 & 124.47 & Agrinio & URANUS & 2012-2015\\
AMAL & 37.7959 & 21.3554 & 99.52 & Amaliada & URANUS & 2012-2015\\
ART2 & 39.1529 & 20.9333 & 49.62 & Arta & URANUS  & 2012-2014\\
EYPA & 38.4268 & 21.9284 & 196.83 & Efpalio & CRLAB & 2002-2015\\
GARG & 37.0662 & 21.6398 & 313.01 & Gargalianoi & URANUS  & 2013-2015\\
HGOU & 39.4927 & 20.2679 & 56.07 & Hgoumenitsa & URANUS  & 2013-2015\\
IOAN & 39.6549 & 20.8541 & 559.07 & Ioannina & URANUS & 2012-2015\\
KASI & 39.7464 & 19.9355 & 108.88 & Kassiopi & NOANET & 2007-2015\\
KEFA & 38.1957 & 20.4382 & 44.51 & Lixouri & URANUS  & 2012-2015\\
KIPO & 38.2032 & 20.3484 & 128.33 & Kipouria & NOANET & 2010-2015\\
KOUN & 38.2095 & 22.0458 & 594.59 & Kounina & CRLAB & 2003-2015\\
LEUK & 38.8299 & 20.7084 & 38.87 & Lefkada & URANUS  & 2012-2015\\
PAT0 & 38.2837 & 21.7868 & 121.05 & Patra & EPN& 2009-2015\\
PONT & 38.6190 & 20.5852 & 48.86 & Ponti & NOANET & 2007-2015\\
RLSO & 38.0558 & 21.4647 & 132.90 & Riolos & NOANET & 2006-2015\\
SPAN & 38.7813 & 20.6736 & 451.34 & Spanochori & NOANET & 2007-2015\\
TROP & 37.7301 & 21.9595 & 774.87 & Tropaia & URANUS  & 2013-2015\\
VASS & 37.4304 & 21.8993 & 1175.53 & Vasses & COMET & 2004-2011\\
VLSM & 38.1768 & 20.5886 & 437.22 & Valsammata & NOANET & 2006-2015\\
PAXI$^1$ & 39.2356 & 20.1282 & 85.00 & Paxi  & HELNET &  1993-2003\\
SKIN$^1$ & 37.9309 & 20.7022 & 87.00 & Skinari & HELNET & 1993-2003\\
KERI$^1$ & 37.6547 & 20.8081 & 214.00 & Keri & HELNET & 1993-2003\\
STRF$^1$ & 37.2501 & 21.0142 & 52.00 & Strofades & HELNET & 1993-2003 \\
\bottomrule
\multicolumn{7}{l}{NOANET: Permanent GNSS network of NOA-GI \citep{Ganas2008}}\\
\multicolumn{7}{l}{CRLAB: Permanent GNSS network on the Corinth's Gulf (\url{http://crlab.eu/})}\\
\multicolumn{7}{l}{EPN: European Permanent Network \citep{Bruyninx2012}}\\
\multicolumn{7}{l}{URANUS: Permanent GNSS network of Tree-Company SA.}\\
\multicolumn{7}{l}{HELNET: Permanent GNSS network of ETHZ \citep{Hollenstein2008}}\\
\multicolumn{7}{l}{$^1$ Station velocities have been recovered from \citet{Hollenstein2008, Hollenstein2003}}\\

   \end{tabular*}
 \end{center}}
\end{table}


\begin{figure}[H]
  \begin{center}
    \includegraphics[width=\linewidth]{ionioNET.jpg}\par
     \caption{ Distribution of permanent and campaign sites for the Western Greece.}
    \label{fig_e:ionioNET}
  \end{center}
\end{figure}


\begin{table}[H]{\small
    \caption{Sites of campaign GPS measurements. SING Network.}
    \label{tab_e:ioncsta}
     \begin{center}
      \begin{tabular*}{\linewidth}{@{\extracolsep{\fill}}l c c r c c}
\toprule

Κωδ. & φ ($^{\circ}$) & λ ($^{\circ}$) & \multicolumn{1}{c}{h (m)} & Epochs of & Time period \\ 
     & & & & Observations & \\
\midrule

AGHL & 38.4831 & 21.2884 &  240.37 & 3 & 1998-2001\\
AGRP & 38.3947 & 21.7403 &  537.04 & 3 & 1995-2001\\
ARIO & 38.3271 & 21.7659 &  35.76 & 3 & 1998-2001\\
ATKO & 38.4940 & 21.1198 &  163.87 & 3 & 1998-2001 \\
CG13 & 38.9053 & 21.7984 &  1012.3 & 8 & 1989-2000\\
CG30 & 38.3966 & 22.1415 &  890.75 & 5 & 1991-2000\\
CG42 & 38.2276 & 21.9728 &  753.83 & 8 & 1989-2000\\
CG61 & 38.0137 & 21.5795 &  287.19 & 5 & 1989-1998\\
DOXA & 37.7024 & 21.9251 &  625.84 & 5 & 1989-2000\\
DREP & 38.3389 & 21.8494 &  33.59 & 6 & 1991-2001\\
E000 & 38.1911 & 22.1021 &  1017.90 & 11 & 1990-2001\\
G000 & 38.0656 & 21.9502 &  786.79 & 9 & 1990-2001\\
H000 & 38.5287 & 21.9803 &  599.42 & 5 & 1991-2001\\
I000 & 38.4447 & 21.9029 &  553.14 & 7 & 1991-2001\\
K000 & 38.2570 & 21.8892 &  1079.70 & 9  & 1990-2001\\
KALK$^1$&38.1640&21.4831 & 92.80 & 3 & 1998-2000\\
KAST & 37.8904 & 21.1416 &  268.86 & 7 & 1989-2000\\
KYLN & 37.9406 & 21.1363 &  47.93 & 3 & 1998-2000\\
L000 & 38.1046 & 21.8079 &  588.77 & 5 & 1991-2001\\
LEPE & 38.6966 & 21.2914 &  167.57 & 4 & 1991-2000\\
LEVK & 38.6072 & 22.1972 &  683.50 & 3 & 1998-2000\\
MAOR & 38.1861 & 21.3896 &  73.35 & 3 & 1998-2000\\
MESS & 38.3701 & 21.1223 &  26.37 & 5 & 1989-2000\\
MLSS & 37.9455 & 21.3520 &  119.48 & 3 & 1998-2000\\
MYRT$^1$&38.0747& 21.5043& 153.89 & 3 & 1998-2000\\
PRSL & 38.6552 & 21.4864 &  716.96 & 3 & 1998-2000\\
RION & 38.3114 & 21.7802 &  34.23 & 3 & 1999-2000\\
RIOE$^2$&38.3114& 21.7802& 35.37 & 3 & 1998-1999\\
THOM$^1$&38.3920&21.4878&122.4857& 3 & 1998-2000\\
TOLO & 38.3240 & 22.1840 & 131.74 & 15 & 1990-2001\\
VELT & 37.9367 & 21.5395 & 362.01 & 3 & 1998-2000\\

\bottomrule
\multicolumn{6}{l}{$^1$ The results of these stations were not used in the analysis.}\\
\multicolumn{6}{l}{$^2$ Eccentric site of RION.}\\
   \end{tabular*} 
 \end{center}}
\end{table}


\begin{figure}[H]
  \begin{center}
    \includegraphics[width=.9\linewidth]{patraNET.jpg}
     \caption{Distribution of permanent and campaign sites for the Central Ionian Sea and Patra's Gulf region.}
    \label{fig_e:patraNET}
  \end{center}
\end{figure}


% ***************** CHAPTER 5 ******************************************************



\begin{figure}[H]
  \begin{center}
    \includegraphics[width=\linewidth]{trianglesALL1.jpg}
    \caption{Triangulation network and the triangles formed over the entire area of Greece. The number in the center of each triangle is the unique code that is used in processing.}
    \label{fig_e:trianglesALL}
  \end{center}
\end{figure}

\begin{figure}[H]
  \begin{center}
    \includegraphics[width=\linewidth]{sterea0201.jpg}
    \caption{Directions of extension (red) and of maximum shear strain (blue) in the region of Eastern Sterea and Gulf of Saronikos for the second period of processing (1930 - 1949/68). The maximum shear direction (blue) represents also the total shear rate ($\dot{\gamma}_{tot}$) size for each triangle.}
    \label{fig_e:sterea0201}
  \end{center}
\end{figure}

\begin{figure}[H]
  \begin{center}
    \includegraphics[width=\linewidth]{sterea0301.jpg}
    \caption{Directions of extension (red) and of maximum shear strain (blue) in the region of Eastern Sterea and Gulf of Saronikos for the second period of processing (1949/68 - 1969/87) for the triangles with high rate of total shear. The maximum shear direction (blue) represents also the total shear rate ($\dot{\gamma}_{tot}$) size for each triangle.}
    \label{fig_e:sterea0301}
  \end{center}
\end{figure}

\begin{figure}[H]
  \begin{center}
    \includegraphics[width=\linewidth]{ext_ionio.jpg}
    \caption{Principal axes of extension for the three periods of terrestrial observations analysis in the Western Greece.}
    \label{fig_e:ext_ionio}
  \end{center}
\end{figure}

\begin{figure}[H]
  \begin{center}
    \includegraphics[width=\linewidth]{shear_ionio.jpg}
    \caption{Direction and magnitude of total shear rate for the three periods of terrestrial observations analysis in the Western Greece.}
    \label{fig_e:shear_ionio}
  \end{center}
\end{figure}

\begin{figure}[H]
  \begin{center}
    \includegraphics[width=\linewidth]{ext_sterea.jpg}
    \caption{Principal axes of extension for the three periods of terrestrial observations analysis in the region of Eastern Sterea.}
    \label{fig_e:ext_sterea}
  \end{center}
\end{figure}

\begin{figure}[H]
  \begin{center}
    \includegraphics[width=\linewidth]{shear_sterea.jpg}
    \caption{Direction and magnitude of total shear rate for the three periods of terrestrial observations analysis in the region of Eastern Sterea.}
    \label{fig_e:shear_sterea}
  \end{center}
\end{figure}




% ***************** CHAPTER 6 ******************************************************

\begin{figure}[H]
  \begin{center}
    \includegraphics[width=\linewidth]{ionioVEL.jpg}
    \caption{GPS velocities of permanent stations with respect to a fixed Europe for the Ionian Sea region.}
    \label{fig_e:ionioVEL}
  \end{center}
\end{figure}

\begin{figure}[H]
  \begin{center}
    \includegraphics[width=\linewidth]{patraVEL.jpg}
    \caption{GPS velocities with respect to a fixed Europe of permanent (blue) and campaign (red) sites for the central Ionian Sea and Patras Gulf region.}
    \label{fig_e:patraVEL}
  \end{center}
\end{figure}


\begin{figure}[H]
  \begin{center}
    \includegraphics[width=\linewidth]{profvel4003.jpg}
    \caption{GPS velocity components along (blue) and across (red) to a coordinate system rotated by 40$^{\circ}$. }
    \label{fig_e:profvel40}
  \end{center}
\end{figure}

\begin{figure}[H]
  \begin{center}
    \includegraphics[width=\linewidth]{iontrSTRcmt.jpg}
    \caption{Principal axes ($\dot{e}_{max}$, $\dot{e}_{min}$) of strain rates ellipses for each triangle of the Ionian Sea region and respective focal mechanisms of significant earthquakes \citep{harvcmt02,Papazachos1997}.}
    \label{fig_e:iontrSTRcmt}
  \end{center}
\end{figure}

\begin{figure}[H]
  \begin{center}
    \includegraphics[width=\linewidth]{ion4bSTRcmt.jpg}
    \caption{Principal axes ($\dot{e}_{max}$, $\dot{e}_{min}$) of strain rates ellipses for the four blocks model of the Western Greece region and focal mechanisms of significant earthquakes \citep{harvcmt02,Papazachos1997}.}
    \label{fig_e:ion4bSTRcmt}
  \end{center}
\end{figure}

\begin{figure}[H]
  \begin{center}
    \includegraphics[width=\linewidth]{ion4bROT.jpg}
    \caption{Rotational rates ($\dot{\Omega}$) of each block for the Western Greece region.}
    \label{fig_e:ion4bROT}
  \end{center}
\end{figure}

\begin{figure}[H]
  \begin{center}
    \includegraphics[width=\linewidth]{ptr05bSTR.jpg}
    \caption{Principal axes ($\dot{e}_{max}$, $\dot{e}_{min}$) of strain rates ellipses  of the five blocks model for the Cetnral Ionian.}
    \label{fig_e:ptr05bSTR}
  \end{center}
\end{figure}

\begin{figure}[H]
  \begin{center}
    \includegraphics[width=\linewidth]{ptr05bROT.jpg}
    \caption{Rotational rates ($\dot{\Omega}$) of the five blocks model for the Central Ionian.}
    \label{fig_e:ptr05bROT}
  \end{center}
\end{figure}

\begin{figure}[H]
  \begin{center}
    \includegraphics[width=\linewidth]{ciontrSTR.jpg}
    \caption{Principal axes ($\dot{e}_{max}$, $\dot{e}_{min}$) of strain rates of the sixteen blocks model for the Western Greece. Dashed green lines represent the triangles of terrestrial measurements (Chapter \ref{chapter5}). Gray lines refer to the GPS blocks.}
    \label{fig_e:ciontrSTR}
  \end{center}
\end{figure}

% ***************** CHAPTER 7 ******************************************************

\begin{table}[H]{\small
    \caption{Fault geometry of the chosen earthquakes \citep{Comninakis1986,Papazachos2003,Makropoulos2012,Shaw2010}.}
    \label{tab_e:fm1900}
     \begin{center}
      \begin{tabular*}{\linewidth}{@{\extracolsep{\fill}}r c c c c c c c c c c}
\toprule

region & date & φ & λ & \multicolumn{1}{c}{Depth} & 
\multicolumn{1}{c}{M\textsubscript{w}} & \multicolumn{3}{c}{Mechanism} & 
\multicolumn{1}{c}{L} & \multicolumn{1}{c}{w}\\ 
\cline{7-9}
 & & \multicolumn{2}{c}{(deg)} & \multicolumn{1}{c}{(km)}& & 
\multicolumn{1}{c}{Strike($^{\circ}$)} & \multicolumn{1}{c}{Dip($^{\circ}$)}& 
\multicolumn{1}{c}{Rake($^{\circ}$)}& \multicolumn{2}{c}{(km)} \\ 

\midrule
Ναύπακτος & 1909 & 38.39 & 21.94 & 8 & 6.2 & 85 & 40 &-125 & 17 & 11 \\
Κατουνα   & 1921 & 38.80 & 21.10 & 8 & 6.0 & 195& 60 & 135 & 14 & 10 \\
Κυλλήνη   & 1939 & 37.8  & 21.17 & 8 & 6.3 & 37 & 89 & 176 & 26 &  9 \\
Κατούνα   & 1953 & 38.80 & 21.10 & 8 & 6.3 & 195& 60 & 135 & 20 & 12 \\
Κεφαλονιά & 1953 & 38.13 & 20.74 & 10 & 7.2 & 163 & 34 & 101 & 55 & 21 \\
Ελίκη     & 1965 & 38.25 & 22.07 & 8 & 6.3 & 290& 30 & -79 & 20 & 12 \\
 
\bottomrule
   \end{tabular*}
 \end{center}}
\end{table}

\begin{table}[H]{\small
    \caption{Fault plane solutions for January 26th, 2014 earthquake from three different studies.}
    \label{tab_e:cmt2601}
     \begin{center}
      \begin{tabular*}{\linewidth}{@{\extracolsep{\fill}}c c c c c c c c c c}
\toprule

code & φ & λ & \multicolumn{1}{c}{Depth} & \multicolumn{3}{c}{Mechanism} & L & w 
& ref.\\ 
\cline{5-7}
 &  \multicolumn{2}{c}{(deg)} & \multicolumn{1}{c}{(km)}& 
\multicolumn{1}{c}{Strike($^{\circ}$)} & \multicolumn{1}{c}{Dip($^{\circ}$)}& 
\multicolumn{1}{c}{Rake($^{\circ}$)} & \multicolumn{2}{c}{(km)} & \\ 

\midrule
260101 & 38.2100 & 20.4610 & 13.5 & 18 & 67 & 164 & 13.2 & 7.9 & 1\\
260102 & 38.1944 & 20.3519 & 15.0 & 20 & 74 & 163 & 25.0 & 10.0 & 2\\
260103 & 38.2133 & 20.4772 & 10.0 & 24 & 72 & 169 & 20.0 & 10.0 & 3\\
\bottomrule
\multicolumn{10}{l}{\footnotesize{References: 1:\citet{Ganas2015} , 2: 
\citet{Sokos2015}, 3: \citet{Sakkas2015} }}\\
   \end{tabular*}
 \end{center}}
\end{table}

\begin{table}[H]{\small
    \caption{Fault plane solutions for the February 3rd, 2014 earthquake from two 
different studies.}
    \label{tab_e:cmt0302}
     \begin{center}
      \begin{tabular*}{\linewidth}{@{\extracolsep{\fill}}c c c c c c c c c c}
\toprule

code & φ & λ & \multicolumn{1}{c}{Depth} & \multicolumn{3}{c}{Mechanism} & L & w 
& ref.\\ 
\cline{5-7}
 &  \multicolumn{2}{c}{(deg)} & \multicolumn{1}{c}{(km)}& 
\multicolumn{1}{c}{Strike($^{\circ}$)} & \multicolumn{1}{c}{Dip($^{\circ}$)}& 
\multicolumn{1}{c}{Rake($^{\circ}$)} & \multicolumn{2}{c}{(km)} & \\ 

\midrule
030201 & 38.2845 & 20.4089 & 10.0 & 199& 49 & 167 & 25.0 & 10.0 & 1\\
\multirow{2}{*}{030203}& 38.3460 & 20.4270 & 10.0 & 180& 86 & 147 & 24.0 & 10.0 
& 2\\
& 38.1970 & 20.3710 & 10.0 & 33 & 76 & 164 & 10.0 & 10.0 & 2\\
\bottomrule
\multicolumn{10}{l}{\footnotesize{References: 1:\citet{Sokos2015}, 2: 
\citet{Boncori2015} }}\\
   \end{tabular*}
 \end{center}}
\end{table}

% NAFPAKTOS
\begin{figure}[H]
\centering
\begin{subfigure}{.5\textwidth}
  \centering
      \includegraphics[width=\linewidth]{naf_1909sh1.jpg}
      \caption{}
      \label{fig_e:naf_1909sh1}
\end{subfigure}

\begin{subfigure}{.49\textwidth}
  \centering
      \includegraphics[width=\linewidth]{naf_1909disp.jpg}
      \caption{}
      \label{fig_e:naf_1909disp}
\end{subfigure}
\begin{subfigure}{.49\textwidth}
\centering
      \includegraphics[width=\linewidth]{naf_1909str.jpg}
      \caption{}
      \label{fig_e:naf_1909str}
\end{subfigure}
\caption{Analysis of the eartquake of Nafpaktos (1909). (\subref{fig_e:naf_1909sh1}) Map view of the shear stress change field and total shear rates for the first campaign period (1895 - 1930). (\subref{fig_e:naf_1909disp}) Horizontal surface displacements calculated from \citet{Okada1985} dislocation model at triangulation stations. (\subref{fig_e:naf_1909str}) Principal axes (${e}_{max}$, ${e}_{min}$) of strain tensor ellipses. The numbers marked in yellow indicate the magnitude of total  shear ($\gamma_{tot}$) for each triangle in μstrain.}
\label{fig_e:naf_1909}
\end{figure}

% KATOUNA 21
\begin{figure}[H]
\centering
\begin{subfigure}{.5\textwidth}
  \centering
      \includegraphics[width=\linewidth]{kat_1921sh1.jpg}
      \caption{}
      \label{fig_e:kat_1921sh1}
\end{subfigure}

\begin{subfigure}{.49\textwidth}
  \centering
      \includegraphics[width=\linewidth]{kat_1921disp.jpg}
      \caption{}
      \label{fig_e:kat_1921disp}
\end{subfigure}
\begin{subfigure}{.49\textwidth}
\centering
      \includegraphics[width=\linewidth]{kat_1921str.jpg}
      \caption{}
      \label{fig_e:kat_1921str}
\end{subfigure}
\caption{Anaslysis of the earthquake of Katouna (1921). (\subref{fig_e:kat_1921sh1}) Map view of the shear stess change field due to the Katouna earthquake (1921) and total shear rates for the first campaign period (1895 - 1930). (\subref{fig_e:kat_1921disp}) Horizontal surface displacements calculated from \citet{Okada1985} dislocation model at triangulation stations. (\subref{fig_e:kat_1921str}) Principal axes (${e}_{max}$, ${e}_{min}$) of strain tensor ellipses. The numbers marked in yellow indicate the magnitude of total shear ($\gamma_{tot}$) for each triangle in μstrain.}
\label{fig_e:kat_1921}
\end{figure}

% KATOUNA 53
\begin{figure}[H]
\centering
\begin{subfigure}{.5\textwidth}
  \centering
      \includegraphics[width=\linewidth]{kat_1953sh2.jpg}
      \caption{}
      \label{fig_e:kat_1953sh2}
\end{subfigure}

\begin{subfigure}{.49\textwidth}
  \centering
      \includegraphics[width=\linewidth]{kat_1953disp.jpg}
      \caption{}
      \label{fig_e:kat_1953disp}
\end{subfigure}
\begin{subfigure}{.49\textwidth}
\centering
      \includegraphics[width=\linewidth]{kat_1953str.jpg}
      \caption{}
      \label{fig_e:kat_1953str}
\end{subfigure}
\caption{Analysis of the eartquake of Katouna (1953). (\subref{fig_e:kat_1953sh2}) Map view of the shear stress change field and total shear rates for the second campaign period (1930 - 1949/68). (\subref{fig_e:kat_1953disp}) Horizontal surface displacements calculated from \citet{Okada1985} dislocation model at triangulation stations. (\subref{fig_e:kat_1953str}) Principal axes (${e}_{max}$, ${e}_{min}$) of strain tensor ellipses. The numbers marked in yellow indicate the magnitude of total shear ($\gamma_{tot}$) for each triangle in μstrain.}
\label{fig_e:kat_1953}
\end{figure}

% KYLINI 39
\begin{figure}[H]
\centering
\begin{subfigure}{.49\textwidth}
  \centering
      \includegraphics[width=.9\linewidth]{kyl_1939sh1.jpg}
      \caption{}
      \label{fig_e:kyl_1939sh1}
\end{subfigure}
\begin{subfigure}{.49\textwidth}
\centering
      \includegraphics[width=.9\linewidth]{kyl_1939sh2.jpg}
      \caption{}
      \label{fig_e:kyl_1939sh2}
\end{subfigure}

\begin{subfigure}{.49\textwidth}
  \centering
      \includegraphics[width=.9\linewidth]{kyl_1939disp.jpg}
      \caption{}
      \label{fig_e:kyl_1939disp}
\end{subfigure}
\begin{subfigure}{.49\textwidth}
\centering
      \includegraphics[width=.9\linewidth]{kyl_1939str.jpg}
      \caption{}
      \label{fig_e:kyl_1939str}
\end{subfigure}
\caption{Map view of the shear stress change field due to the Kyllini earthquake (1939). (\subref{fig_e:kyl_1939sh1}) Principal axes of extension (red) and direction of maximum shear strain(black) for the first campaign period (1895 - 1930). (\subref{fig_e:kyl_1939sh2}) Principal axes of extension (red) and direction of maximum shear strain (black) for the second campaign period (1930 - 1949/68). (\subref{fig_e:kyl_1939disp}) Horizontal surface displacements calculated from \citet{Okada1985} dislocation model at triangulation stations. (\subref{fig_e:kyl_1939str}) Principal axes (${e}_{max}$, ${e}_{min}$) of strain tensor ellipses. The numbers marked in yellow indicate the magnitude of total shear ($\gamma_{tot}$) for each triangle in μstrain.}
\label{fig_e:kyl_1939}
\end{figure}

% KEFALONIA 53
\begin{figure}[H]
\centering
\begin{subfigure}{.49\textwidth}
  \centering
      \includegraphics[width=.9\linewidth]{kef_1953sh1.jpg}
      \caption{}
      \label{fig_e:kef_1953sh1}
\end{subfigure}
\begin{subfigure}{.49\textwidth}
\centering
      \includegraphics[width=.9\linewidth]{kef_1953sh2.jpg}
      \caption{}
      \label{fig_e:kef_1953sh2}
\end{subfigure}

\begin{subfigure}{.49\textwidth}
  \centering
      \includegraphics[width=.9\linewidth]{kef_1953disp.jpg}
      \caption{}
      \label{fig_e:kef_1953disp}
\end{subfigure}
\begin{subfigure}{.49\textwidth}
\centering
      \includegraphics[width=.9\linewidth]{kef_1953str.jpg}
      \caption{}
      \label{fig_e:kef_1953str}
\end{subfigure}
\caption{Map view of the shear stress change field due to the Kefalonia earthquake (1953). (\subref{fig_e:kef_1953sh1}) Principal axes of extension (red) and direction of maximum shear strain (black) for the first campaign period (1895 - 1930). (\subref{fig_e:kef_1953sh2}) Principal axes of extension (red) and direction of maximum shear strain (black) for the second campaign period (1930 - 1949/68). (\subref{fig_e:kef_1953disp}) Horizontal surface displacements calculated from \citet{Okada1985} dislocation model at triangulation stations. (\subref{fig_e:kef_1953str}) Principal axes (${e}_{max}$, ${e}_{min}$) of strain tensor ellipses. The numbers marked in yellow indicate the magnitude of total shear ($\gamma_{tot}$) for each triangle in μstrain.}
\label{fig_e:kef_1953}
\end{figure}

% ELIKI 65
\begin{figure}[H]
\centering
\begin{subfigure}{.49\textwidth}
  \centering
      \includegraphics[width=.9\linewidth]{eli_1965sh2.jpg}
      \caption{}
      \label{fig_e:eli_1965sh2}
\end{subfigure}
\begin{subfigure}{.49\textwidth}
\centering
      \includegraphics[width=.9\linewidth]{eli_1965sh3.jpg}
      \caption{}
      \label{fig_e:eli_1965sh3}
\end{subfigure}

\begin{subfigure}{.49\textwidth}
  \centering
      \includegraphics[width=.9\linewidth]{eli_1965disp.jpg}
      \caption{}
      \label{fig_e:eli_1965disp}
\end{subfigure}
\begin{subfigure}{.49\textwidth}
\centering
      \includegraphics[width=.9\linewidth]{eli_1965str.jpg}
      \caption{}
      \label{fig_e:eli_1965str}
\end{subfigure}
\caption{Map view of the shear stress change due to the Eliki 
earthquake (1965). (\subref{fig_e:eli_1965sh2}) Principal axes of extension (red) 
and direction of maximum shear strain (black) for the secong campaign period 
(1930 - 1949/68). (\subref{fig_e:eli_1965sh3}) Principal axes of extension (red) 
and direction of maximum shear strain (black) for the third campaign period 
(1949/68 - 1969/87). (\subref{fig_e:eli_1965disp}) Horizontal surface 
displacements calculated from \citet{Okada1985} dislocation model at 
triangulation stations. (\subref{fig_e:eli_1965str}) Principal axes (${e}_{max}$, 
${e}_{min}$) of strain tensor ellipses. The numbers marked in yellow indicate 
the magnitude of total shear ($\gamma_{tot}$) for each triangle in μstrain.}
\label{fig_e:eli_1965}
\end{figure}

\begin{figure}[H]
  \begin{center}
    \includegraphics[width=.95\linewidth]{tsall.png}
    \caption{Time series analysis of six permanent GNSS stations. Green line depicts the linear adjustment model for each time series.}
    \label{fig_e:tsallkefa}
  \end{center}
\end{figure}

\begin{figure}[H]
\centering
\begin{subfigure}{.49\textwidth}
  \centering
      \includegraphics[width=\linewidth]{ts_kefa_1m.png}
      \caption{}
      \label{fig_e:ts_kefa_1m}
\end{subfigure}
\begin{subfigure}{.49\textwidth}
\centering
      \includegraphics[width=\linewidth]{ts_vlsm_1m.png}
      \caption{}
      \label{fig_e:ts_vlsm_1m}
\end{subfigure}
\caption{Time series analysis for stations KEFA (\subref{fig_e:ts_kefa_1m}) and VLSM (\subref{fig_e:ts_vlsm_1m}) for the time period of one month before and after the two earthquakes. Red lines depict the time of the earthquake occurence.}
\label{fig_e:ts_1m}
\end{figure}

\begin{figure}[H]
\centering
\begin{subfigure}{.49\textwidth}
  \centering
      \includegraphics[width=\linewidth]{ts_3d_2601kefa.png}
      \caption{}
      \label{fig_e:ts_3d_2601kefa}
\end{subfigure}
\begin{subfigure}{.49\textwidth}
\centering
      \includegraphics[width=\linewidth]{ts_3d_2601vlsm.png}
      \caption{}
      \label{fig_e:ts_3d_2601vlsm}
\end{subfigure}
    \caption{Displacements of stations KEFA (\subref{fig_e:ts_3d_2601kefa}) and VLSM (\subref{fig_e:ts_3d_2601vlsm}) for a day before and after the eathquake of January 26th, 2014. GNSS data for the day of the earthquake (DOY=026) have been devided into two parts, before and after the earthquake, and have been processed separately.}
\label{fig_e:tssep}
\end{figure}

\begin{table}[H]{\small
    \caption{Co-seismic displacements for GNSS stations KEFA, VLSM, KIPO.}
    \label{tab_e:kefshift}
     \begin{center}
      \begin{tabular*}{\linewidth}{@{\extracolsep{\fill}} c c c c c c c c }
\toprule

&	& \multicolumn{3}{c|}{January 26th} & \multicolumn{3}{|c}{February 3rd}  \\
\cline{3-8}
\multirow{2}{*}{Station} & \multirow{2}{*}{Type of solution}	& dNorth & dEast 
& dUp & dNorth & dEast & dUp \\
&	& \multicolumn{6}{c}{(mm)} \\
\hline
\multirow{3}{*}{KEFA} &daily$^a$     & -54.7 & 26.2 & 32.2 & -91.0 & 31.0 & 33.0 
\\
& 2-periods$^b$ & -47.8 & 21.7 & 24.5 &   -   &  -    &   -   \\
& PPP$^c$        & -47.0 & 25.0 &  -   & 200.0 & 0    & 80.0 \\
\hline
\multirow{2}{*}{VLSM} & daily$^a$      &  -7.8 & -18.6 & -9.8 & -9.0 & -10.0 & 
2.0 \\
& 2-periods$^b$ & -7.6 & -19.6 & -8.4 &   -    &   -   &  -    \\
\bottomrule
\multirow{2}{*}{KIPO} &  & \multicolumn{3}{c}{both earthquakes}  & 
\multicolumn{3}{c}{} \\
\cline{3-5}
& daily$^a$ & 80.2 & -15.4 & 52.4 & \multicolumn{3}{c}{} \\
\bottomrule
\multicolumn{8}{l}{\footnotesize{$^a$ Daily solution.}}\\
\multicolumn{8}{l}{\footnotesize{$^b$ Seperate the day of the earthquake}}\\
\multicolumn{8}{l}{\footnotesize{$^c$ Processing 1Hz GNSS data with Precise Point Positioning(PPP).}}\\
    \end{tabular*}
  \end{center}}
\end{table}

\begin{figure}[H]
  \begin{center}
    \includegraphics[width=\linewidth]{disp_split.jpg}
    \caption{Co-seismic deisplacements at the GNSS stations due to the earthquakes of 26th of January (blue arrows) and 3rd of February (red arrows).}
    \label{fig_e:disp_split}
  \end{center}
\end{figure}

\begin{figure}[H]
  \begin{center}
    \includegraphics[width=\linewidth]{disp_merge.jpg}
    \caption{Total displacement of the GNSS stations for the period since before the 26th of January until after the 3rd of February.}
    \label{fig_e:disp_merge}
  \end{center}
\end{figure}

\begin{figure}[H]
\centering
\begin{subfigure}{.49\textwidth}
  \centering
      \includegraphics[width=.95\linewidth]{inp_260101h.jpg}
      \caption{}
      \label{fig_e:inp_260101h}
\end{subfigure}
\begin{subfigure}{.49\textwidth}
\centering
      \includegraphics[width=.95\linewidth]{inp_260101v.jpg}
      \caption{}
      \label{fig_e:inp_260101v}
\end{subfigure}
\caption{Horizontal (\subref{fig_e:inp_260101h}) and vertical (\subref{fig_e:inp_260101v}) surface displacements regarding the January 26th earthquake. Blue  arrows dipict displacements calulated from Okada dislocation model using fault plane solution after \citet{Ganas2015} and red arrows present the observed displacements of GNSS stations.}
\label{fig_e:inp_260101}
\end{figure}

\begin{figure}[H]
\centering
\begin{subfigure}{.49\textwidth}
  \centering
      \includegraphics[width=.95\linewidth]{inp_030203h.jpg}
      \caption{}
      \label{fig_e:inp_030203h}
\end{subfigure}
\begin{subfigure}{.49\textwidth}
\centering
      \includegraphics[width=.95\linewidth]{inp_030203v.jpg}
      \caption{}
      \label{fig_e:inp_030203v}
\end{subfigure}
\caption{Horizontal (\subref{fig_e:inp_030203h}) and vertical (\subref{fig_e:inp_030203v}) surface displacements regarding the February 3rd earthquake. Surface displacements calulated from Okada dislocation model used fault plane solution after \citet{Boncori2015}.}
\label{fig_e:inp_030203}
\end{figure}

% ***************** CHAPTER 8 ******************************************************






%% ********************************** Bibliography ******************************
%\begin{spacing}{0.9}
%
%% To use the conventional natbib style referencing
%% Bibliography style previews: http://nodonn.tipido.net/bibstyle.php
%% Reference styles: http://sites.stat.psu.edu/~surajit/present/bib.htm
%
%\bibliographystyle{apalike}
%%\bibliographystyle{unsrt} % Use for unsorted references  
%% \bibliographystyle{plainnat} % use this to have URLs listed in References
%\cleardoublepage
%\bibliography{References/triangleref} % Path to your References.bib file
%
%% If you would like to use BibLaTeX for your references, pass `custombib' as
%% an option in the document class. The location of 'reference.bib' should be
%% specified in the preamble.tex file in the custombib section.
%% Comment out the lines related to natbib above and uncomment the following line.
%
%% \bibliography[heading=bibintoc, title={References}]
%
%\end{spacing}














\end{extsum}
%\end{otherlanguage}
