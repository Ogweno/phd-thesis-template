% ************************ Thesis Information & Meta-data **********************
%% The title of the thesis
\title{ΠΡOΤΥΠΟ ΚΕΙΜΕΝΟΥ \LaTeX ,  \\ΔΙΔΑΚΤΟΡΙΚΗΣ ΔΙΔΑΤΡΙΒΗΣ ΤΟΥ Ε.Μ.Π.}
\titleen{PhD THESIS \LaTeX TEMPLATE \\ FOR NTUA}

%% Subtitle (Optional)
\subtitle{\textcolor{red}{ΔΕΝ ΑΠΟΤΕΛΕΙ ΕΠΙΣΗΜΟ ΕΓΓΡΑΦΟ ΤΟΥ Ε.Μ.Π.}}
\subtitleen{\textcolor{red}{NOT AN OFFICIAL TAMPLATE OF NTUA}}

%\texorpdfstring is used for PDF metadata. Usage:
%\texorpdfstring{LaTeX_Version}{PDF Version (non-latex)} eg.,
%\texorpdfstring{$sigma$}{sigma}

%% The full name of the author in genetive
\authortou{ΔΗΜΗΤΡΙΟΥ Γ. ΑΝΑΣΤΑΣΙΟΥ}
%% The full ne name of the author
\author{ΔΗΜΗΤΡΙΟΣ Γ. ΑΝΑΣΤΑΣΙΟΥ}
\authoren{Demitris G. Anastasiou}

\diploma{Διπλωματούχου Αγρονόμου Τοπογράφου Μηχανικού Ε.Μ.Π.}
\diplomaen{Dipl. Rural \& Surveying Engineer NTUA}

%% School (eg. Department of Engineering, Maths, Physics)
\dept{ΣΧΟΛΗ ΑΓΡΟΝΟΜΩΝ \& ΤΟΠΟΓΡΑΦΩΝ ΜΗΧΑΝΙΚΩΝ}

%% University and Crest
\university{ΕΘΝΙΚΟ ΜΕΤΣΟΒΙΟ ΠΟΛΥΤΕΧΝΕΙΟ}

% Crest minimum should be 30mm.
\crest{\includegraphics[width=0.98\textwidth]{ntua.png}}
%% Use this crest, if you are using the college crest
%% Crest long miminum should be 65mm
%\crest{\includegraphics[width=0.45\textwidth]{University_Crest_Long}}

%% College shield [optional] 
% Crest minimum should be 30mm.
% \collegeshield{\includegraphics[width=0.2\textwidth]{CollegeShields/Kings}}

%% You can redefine the submission text:
% Default as per the University guidelines:
% ``This dissertation is submitted for the degree of''
% \renewcommand{\submissiontext}{}

%% Full title of the Degree
\degreetitle{ΔΙΔΑΚΤΟΡΙΚΗ ΔΙΑΤΡΙΒΗ}

%supervisor
\supervisor{...............\\Καθηγητής Ε.Μ.Π.}
%advisors
\advisora{........., Καθ. Ε.Μ.Π.}
\advisorb{........., τ. Καθ. Ε.Μ.Π.}
\advisorc{........., τ. Καθ. Ε.Μ.Π.}
%examiners
\examinerd{........., Καθ. Ε.Μ.Π.}
\examinere{........., Αναπλ. Καθ. Ε.Μ.Π.}
\examinerf{........., Διευθ. Ερευνών Ε.Α.Α.}
\examinerg{........., Διευθ. Ερευνών Ε.Α.Α.}


%% College location
\city{ΑΘΗΝΑ}

%% Submission date
% Default is set as {\monthname[\the\month]\space\the\year}
\degreedate{Ιούνιος 2017} 

%% Meta information
\subject{Γεωδαισία} \keywords{{Γεωδαισία} {Τριγωνισμός} {Παραμόρφωση} {Ελλάδα}}
