%!TEX root = ../thesis.tex
%*******************************************************************************
%****************************** Second Chapter *********************************
%*******************************************************************************

\chapter{Διαμόρφωση Κειμένου}

\ifpdf
    \graphicspath{{Chapter2/Figs/Raster/}{Chapter2/Figs/PDF/}{Chapter2/Figs/}}
\else
    \graphicspath{{Chapter2/Figs/Vector/}{Chapter2/Figs/}}
\fi

%********************************** %First Section  **************************************

\section[Συμπτηγμένος τίτλος]{Μεγάλος τίτλος ενός κεφαλαίου}

% Uncomment this line, when you have siunitx package loaded.
%The SI Units for dynamic viscosity is \si{\newton\second\per\metre\squared}.
Αναφορά σε μια εικόνα Σχήμα~\ref{fig:minion} ή στον Πίνακα \ref{table:nice_table}.



%********************************** %Second Section  *************************************

\section{Μορφή κειμένου}
\textbf{Χρήση έντονης γραμματοσειράς\index{γραμματοσειρά}.}

\textit{Χρήση πλάγιων γραμμάτων.}

\textcolor{red}{Μπορούν να χρησιμοποιηθούν διαφορετικά χρώματα\index{χρώματα} για την σημείωση κάποιων τμημάτων του κειμένου.}\colorbox{BurntOrange}{είτε να τονιστούν με διαφορετικά χρώματα.}

\begin{flushleft}
Στοίχιση του κειμένου αριστερά.
\end{flushleft}

\begin{center}
Στοίχιση του κειμένου στο κέντρο της σελίδας.
\end{center}

\begin{flushright}
Στοίχιση κειμένου στα δεξιά της σελίδας.
\end{flushright}


\section*{enumerate\index{enumerate}}
Οι λίστες μπορούν να εισαχθούν με διαφορετικούς τρόπους όπως δίνονται μερικά παραδείγματα πιο κάτω.

Χρησιμοποιώντας αριθμούς:
\begin{enumerate}
\item The first topic is dull
\item The second topic is duller
\begin{enumerate}
\item The first subtopic is silly
\item The second subtopic is stupid
\end{enumerate}
\item The third topic is the dullest
\end{enumerate}


\section*{Itemize\index{itemize}}
Χρησιμοποιώντας σύμβολα:
\begin{itemize}
\item The first topic is dull
\item The second topic is duller
\begin{itemize}
\item The first subtopic is silly
\item The second subtopic is stupid
\end{itemize}
\item The third topic is the dullest
\end{itemize}

\section*{Description\index{description}}
\begin{description}
\item[The first topic] is dull
\item[The second topic] is duller
\begin{description}
\item[The first subtopic] is silly
\item[The second subtopic] is stupid
\end{description}
\item[The third topic] is the dullest
\end{description}

%********************************** %Third Section  *************************************

\section{Αναφορές στη βιβλιογραφία\index{βιβλιογραφία}}
Αναφορές σε παρενθέσεις \citep{Gam90}, ή μέσα στο κείμενο \citet{Dou72}.

Παραπάνω από μια αναφορά μαζί \citep{Ancey1996, Enf87, Sim94, Sim96b}


%********************************** %Fourth Section  *************************************
\section{Εξισώσεις}

Η πιο δημοφιλής εξίσωση\index{εξίσωση} στον κόσμο: $E^2 = (m_0c^2)^2 + (pc)^2$, γνωστή ως σχέση  \textbf{ενέργειας-μάζας-ορμής} τοποθετημένη μέσα στο κείμενο.

Ή τοποθετημενη ως εξίσωση ξεχωριστά από το κείμενο:
\begin{equation}
\label{eq:enmm}
  E^2 = (m_0c^2)^2 + (pc)^2
\end{equation}

και άλλη μια εξίσωση:

\begin{align}
\label{eq:new}
CIF: \hspace*{5mm}F_0^j(a) = \frac{1}{2\pi \iota} \oint_{\gamma} \frac{F_0^j(z)}{z - a} dz
\end{align}

\nomenclature[z-cif]{$CIF$}{Cauchy's Integral Formula}                                % first letter Z is for Acronyms 
\nomenclature[a-F]{$F$}{complex function}                                                   % first letter A is for Roman symbols
\nomenclature[g-p]{$\pi$}{ $\simeq 3.14\ldots$}                                             % first letter G is for Greek Symbols
\nomenclature[g-i]{$\iota$}{unit imaginary number $\sqrt{-1}$}                      % first letter G is for Greek Symbols
\nomenclature[g-g]{$\gamma$}{a simply closed curve on a complex plane}  % first letter G is for Greek Symbols
\nomenclature[x-i]{$\oint_\gamma$}{integration around a curve $\gamma$} % first letter X is for Other Symbols
\nomenclature[r-j]{$j$}{superscript index}                                                       % first letter R is for superscripts
\nomenclature[s-0]{$0$}{subscript index}                                                        % first letter S is for subscripts


\clearpage
