%!TEX root = ../thesis.tex
%*******************************************************************************
%****************************** Second Chapter *********************************
%*******************************************************************************

\chapter{Διαμόρφωση Κειμένου}

\ifpdf
    \graphicspath{{Chapter2/Figs/Raster/}{Chapter2/Figs/PDF/}{Chapter2/Figs/}}
\else
    \graphicspath{{Chapter2/Figs/Vector/}{Chapter2/Figs/}}
\fi


\section[Συμπτηγμένος τίτλος]{Μεγάλος τίτλος ενός κεφαλαίου}

% Uncomment this line, when you have siunitx package loaded.
%The SI Units for dynamic viscosity is \si{\newton\second\per\metre\squared}.
Αναφορά σε μια εικόνα Σχήμα~\ref{fig:minion} ή στον Πίνακα \ref{table:nice_table}.


Σε περίπτωση που υπάρχει κάποιο πρόβλημα με το παρόν έγγραφο μπορείτε να επικοινωνήσετε με τον Δημήτρη στο: \href{mailto:dganastasiou@gmail.com}{dganastasiou@gmail.com} ή ανοίγοντας ένα νέο αίτημα/θέμα στο \url{https://github.com/demanasta/phd-thesis-template/}

\section{Μορφή κειμένου}
\textbf{Χρήση έντονης γραμματοσειράς.}

\textit{Χρήση πλάγιων γραμμάτων.}

\textcolor{red}{Μπορούν να χρησιμοποιηθούν διαφορετικά χρώματα για την σημείωση κάποιων τμημάτων του κειμένου.}\colorbox{BurntOrange}{είτε να τονιστούν με διαφορετικά χρώματα.}

\begin{flushleft}
Στοίχιση του κειμένου αριστερά.
\end{flushleft}

\begin{center}
Στοίχιση του κειμένου στο κέντρο της σελίδας.
\end{center}

\begin{flushright}
Στοίχιση κειμένου στα αριστερά της σελίδας.
\end{flushright}


\section*{enumerate}
Οι λίστες μπορούν να εισαχθούν με διαφορετικούς τρόπους όπως δίνονται μερικά παραδείγματα πιο κάτω.

Χρησιμοποιώντας αριθμούς:
\begin{enumerate}
\item The first topic is dull
\item The second topic is duller
\begin{enumerate}
\item The first subtopic is silly
\item The second subtopic is stupid
\end{enumerate}
\item The third topic is the dullest
\end{enumerate}


\section*{Itemize}
Χρησιμοποιώντας σύμβολα:
\begin{itemize}
\item The first topic is dull
\item The second topic is duller
\begin{itemize}
\item The first subtopic is silly
\item The second subtopic is stupid
\end{itemize}
\item The third topic is the dullest
\end{itemize}

\section*{Description}
\begin{description}
\item[The first topic] is dull
\item[The second topic] is duller
\begin{description}
\item[The first subtopic] is silly
\item[The second subtopic] is stupid
\end{description}
\item[The third topic] is the dullest
\end{description}


\clearpage


