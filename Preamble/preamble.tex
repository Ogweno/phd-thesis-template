% ******************************************************************************
% ****************************** Custom Margin *********************************

% Add `custommargin' in the document class options to use this section
% Set {innerside margin / outerside margin / topmargin / bottom margin}  and
% other page dimensions
\ifsetCustomMargin
  \RequirePackage[left=37mm,right=30mm,top=35mm,bottom=30mm]{geometry}
  \setFancyHdr % To apply fancy header after geometry package is loaded
\fi

% \usepackage{titling}
% *****************************************************************************
% ******************* Fonts (like different typewriter fonts etc.)*************

% Add `customfont' in the document class option to use this section

\ifsetCustomFont
  % Set your custom font here and use `customfont' in options. Leave empty to
  % load computer modern font (default LaTeX font).
%   \RequirePackage{helvet}

% \setmainfont[Mapping=tex-text]{GFS Didot}
% \setmainfont[Mapping=tex-text]{GFS Bodoni}
% \setmainfont[Mapping=tex-text]{GFS Olga} % είναι λίγο ότι να ναι αυτή!!πλάγια
% \setmainfont[Mapping=tex-text]{GFS Neohellenic}
% \setmainfont[Mapping=tex-text]{GFS Artemisia}
% \setmainfont[Mapping=tex-text]{GFS Elpis} %low resolution printing
% \setmainfont[Mapping=tex-text]{Linux Libertine T}
% \setmainfont[Mapping=tex-text]{Linux Libertine O}

  % For use with XeLaTeX
%    \setmainfont[
%      Path              = ./libertine/opentype/,
%      Extension         = .otf,
%      UprightFont = LinLibertine_R,
%      BoldFont = LinLibertine_RZ, % Linux Libertine O Regular Semibold
%      ItalicFont = LinLibertine_RI,
%      BoldItalicFont = LinLibertine_RZI, % Linux Libertine O Regular Semibold Italic
%    ]
%    {LGR}
  %  % load font from system font
%    \newfontfamily\libertinesystemfont{Linux Libertine O}
\fi

% *****************************************************************************
% **************************** Custom Packages ********************************

% ************************* Algorithms and Pseudocode **************************

%\usepackage{algpseudocode}
% Registerd symbols copyright 
\usepackage{textcomp}
% \textcopyleft \textregistered \textcopyright \sffamily
% MATLAB\textsuperscript \textregistered / Simulink


% ********************Captions and Hyperreferencing / URL **********************

% Captions: This makes captions of figures use a boldfaced small font.
% \RequirePackage[small,bf,labelsep=colon,tableposition=top]{caption}
% \RequirePackage[small,bf,labelsep=colon,tableposition=top]{caption}
\RequirePackage[small,bf,labelsep=space, tableposition=top]{caption}

% \RequirePackage[labelsep=space,tableposition=top]{caption}
% \renewcommand{\figurename}{Εικόνα} %to support older versions of captions.sty


% *************************** Graphics and figures *****************************

\usepackage{rotating}
% \usepackage{wrapfig}

% Uncomment the following two lines to force Latex to place the figure.
% Use [H] when including graphics. Note 'H' instead of 'h'
\usepackage{float}
\restylefloat{figure}

% Subcaption package is also available in the sty folder you can use that by
% uncommenting the following line
% This is for people stuck with older versions of texlive
%\usepackage{sty/caption/subcaption}
\usepackage{subcaption}
\usepackage{xparse}


\usepackage[english, greek]{babel}
%\usepackage[greek,english]{babel}
% \makeatletter
% \newcommand{\dualcap}[2]{
% \caption{#1}
% \addtocounter{\@captype}{-1}\captionsetup{ justification=justified, skip=\z@%%%%%%%%%
% }
% {\selectlanguage{english}\caption{#2}}
% }
% 
% 
% % \newcommand{\dualcap}[2]{{\selectlanguage{greek}
% % \caption{#1}}\addtocounter{\@captype}{-1}\captionsetup{
% % skip=\z@%%%%%%%%%
% % }\caption{#2}
% % }
% \NewDocumentCommand{\encaption}{ o  m }{\addtocounter{\@captype}{-1}{\selectlanguage{english}%
% \IfNoValueTF{#2}{\caption[#1]{#2}}{\caption[#2]{#2}}
% }
% }
% 
% 
% \makeatother
% \usepackage{blindtext}
% \usepackage{libertine}
\usepackage[lang=english,list=off]{bicaption}
\captionsetup[figure][bi]{labelfont=bf, justification=RaggedRight, singlelinecheck=false, format=hang}
\captionsetup[table]{labelfont=bf, justification=RaggedRight, singlelinecheck=false, format=hang}
% \captionsetup[longtabu][bi]{labelfont=bf,  justification=RaggedRight, singlelinecheck=false, format=hang, margin={0mm,0mm}}

% ********************************** Tables ************************************
\usepackage{booktabs} % For professional looking tables
\usepackage{multirow}

\usepackage{multicol}
\usepackage{longtable}
\usepackage{tabularx}
\usepackage{tabu}
\usepackage{adjustbox}

% \usepackage{slashbox}
% \usepackage{threeparttable}
% \usepackage{tablefootnote}
% *********************************** SI Units *********************************
\usepackage{siunitx} % use this package module for SI units


% ******************************* Line Spacing *********************************

% Choose linespacing as appropriate. Default is one-half line spacing as per the
% University guidelines

% \doublespacing
% \onehalfspacing
% \singlespacing


% ************************ Formatting / Footnote *******************************

% Don't break enumeration (etc.) across pages in an ugly manner (default 10000)
%\clubpenalty=500
%\widowpenalty=500

\usepackage[perpage]{footmisc} %Range of footnote options

%\usepackage[symbol*]{footmisc}
%\DefineFNsymbolsTM{myfnsymbols}{% def. from footmisc.sty "bringhurst" symbols
%  \textasteriskcentered *
%  \textdagger    \dagger
%  \textdaggerdbl \ddagger
%  \textsection   \mathsection
%  \textbardbl    \|%
%  \textparagraph \mathparagraph
%}%
%\setfnsymbol{myfnsymbols}



% *****************************************************************************
% *************************** Bibliography  and References ********************

%\usepackage{cleveref} %Referencing without need to explicitly state fig /table

% Add `custombib' in the document class option to use this section
\ifuseCustomBib
   \RequirePackage[round, comma , sort, numbers, authoryear]{natbib} % CustomBib

% If you would like to use biblatex for your reference management, as opposed to the default `natbibpackage` pass the option `custombib` in the document class. 
% Comment out the previous line to make sure you don't load the natbib package. Uncomment the following lines and specify the location of references.bib file

% \RequirePackage[backend=biber, style=numeric-comp, citestyle=numeric, sorting=nty, natbib=true]{biblatex}
% \bibliography{References/triangleref} %Location of references.bib only for biblatex
%
\fi

% changes the default name `Bibliography` -> `References'
\renewcommand{\bibname}{Bιβλιογραφία}
%\renewcommand{\bibname}{Bibliography}


% ******************************** Roman Pages *********************************
% The romanpages environment set the page numbering to lowercase roman one
% for the contents and figures lists. It also resets
% page-numbering for the remainder of the dissertation (arabic, starting at 1).

% \newenvironment{romanpages}{
%   \setcounter{page}{1}
%   \renewcommand{\thepage}{\roman{page}}}
% {\newpage\renewcommand{\thepage}{\arabic{page}}}


% ******************************************************************************
% ************************* User Defined Commands ******************************
% ******************************************************************************

% *********** To change the name of Table of Contents / LOF and LOT ************

\renewcommand{\contentsname}{Περιεχόμενα}
\renewcommand{\listfigurename}{Πίνακας Εικόνων}
\renewcommand{\listtablename}{Πίνακας Πινάκων}


% ********************** TOC depth and numbering depth *************************

\setcounter{secnumdepth}{2}
\setcounter{tocdepth}{2}


% ******************************* Nomenclature *********************************

% To change the name of the Nomenclature section, uncomment the following line

\renewcommand{\nomname}{Πίνακας Συμβόλων, Ακρονυμίων}


% ********************************* Appendix ***********************************

% The default value of both \appendixtocname and \appendixpagename is `Appendices'. These names can all be changed via:

\renewcommand{\appendixtocname}{ερτασγ}
\renewcommand{\appendixname}{ασφωργηηφδ}

% *********************** Configure Draft Mode **********************************
% \usepackage[printwatermark]{xwatermark}
% \newwatermark*[allpages,color=red!50,angle=45,scale=3,xpos=0,ypos=0]{DRAFT}

% Uncomment to disable figures in `draftmode'
%\setkeys{Gin}{draft=true}  % set draft to false to enable figures in `draft'

% These options are active only during the draft mode
% Default text is "Draft"
\SetDraftText{DRAFT}

% Default Watermark location is top. Location (top/bottom)
%\SetDraftWMPosition{bottom}

% Draft Version - default is v1.0
\SetDraftVersion{v1.0}

% Draft Text grayscale value (should be between 0-black and 1-white)
% Default value is 0.7
\SetDraftGrayScale{0.7}

% Set Draft water mark in print mode. Uncomment next lines
% \usepackage{draftwatermark}
% \SetWatermarkText{\parbox{46cm}{%54 
%   * D R A F T - v0.9.7 * \\ \\
%   * \today * \\ \\
%   compiled via \LaTeX}}
% \SetWatermarkScale{.24}%44
% \SetWatermarkColor[rgb]{1,0,0}

% ******************************** Todo Notes **********************************
%% Uncomment the following lines to have todonotes.

\ifsetDraft
  \usepackage[colorinlistoftodos,prependcaption,textsize=small]{todonotes}
  \setlength{\marginparwidth}{2.2cm}
% 	\usepackage[colorinlistoftodos]{todonotes}
	\newcommand{\mynote}[1]{\todo[author=mitsos,size=\small,inline,color=green!40]{#1}}
  \newcommand{\unsure}[1]{\todo[author=mitsos,size=\small,color=red!60]{#1}}
	\newcommand{\change}[2][1=]{\todo[author=mitsos,size=\small,linecolor=blue,backgroundcolor=blue!35,bordercolor=blue]{#1}}
% 	\newcommand{\info}[2][1=]{\todo[linecolor=OliveGreen,backgroundcolor=OliveGreen!25,bordercolor=OliveGreen,#1]{#2}}
% 	\newcommand{\improvement}[2][1=]{\todo[linecolor=Plum,backgroundcolor=Plum!25,bordercolor=Plum,#1]{#2}}
	\newcommand{\reviewa}[1]{\todo[author=reviewa,size=\small,inline,color=red!40]{#1}}
	\newcommand{\reviewb}[1]{\todo[author=reviewb,size=\small,inline,color=red!40]{#1}}
\else
  \newcommand{\todo}[1]{}
	\newcommand{\mynote}[1]{}
	\newcommand{\unsure}[1]{}
	\newcommand{\change}[1]{}
% 	\newcommand{\info}[2][1=]{}
% 	\newcommand{\improvement}[2][1=]{}
	\newcommand{\reviewa}[1]{}
	\newcommand{\reviewb}[1]{}
	\newcommand{\listoftodos}{}
\fi

% Example todo: \mynote{Hey! I have a note}
